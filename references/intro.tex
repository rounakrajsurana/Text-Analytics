\setlength{\footskip}{8mm}

\chapter{Introduction} 

\textit{Introduction would consist of the background, problem statement, objectives, limitations, scope and finally the outline chapter of this report.}

\section{Background}

\text These course evaluations by students often combine computer-graded multiple-choice items with comments. Computer-graded multiple-choice items are easy to assess because the responses are quantifiable. However, the responses to comments are text data, and objectively grasping the students’ general tendencies is challenging. Moreover, it is difficult to avoid risking arbitrary and subjective interpretations of the data by the analysts who summarize them. Therefore, to avoid these risks as much as possible, Sentiment analysis/Text mining  approach might be useful.
\par
Sentiment Analysis is an application of natural language processing, text mining and computational linguistics, to identify information from the text. Education is one of the areas that currently have seen favorable sentiment analysis application, in order to improve attended sessions or distance education. Recent methods for analyzing student course evaluations are manual and it mainly focuses on the quantitative feedback. 


\section{Problem Statement}

Course evaluation and teacher evaluations, where students get to submit their feedback about the course anonymously at the end of each semester. Results of such evaluation forms is most commonly used to improve the course study and as well as receive comments about the course quality, which would rather provide understanding of teaching and course effectiveness. 
Instead of considering the scores their is also need of analyzing the comments so received by the students which would give relevant information about the course and as well as these comments would also help teachers get insight to make proper adjustments to future classes in a much better way.\par 
To solve this problem we build a system to extract important points from the comments so given in the course evaluation, which can add information to the improvement of course study.

\section{Objectives}
The main objective of this research would be text analyzing of the comments given by students in student course evaluation form every semester, for better understanding of student requirements and also identifying the positive comments and the negative comments for the evaluated course. Steps which are required for accomplishing these objectives are: 
\begin{enumerate}
    \item To analyze the text data/comments so given by the students and perform text mining. 

    \item To classify comments based on sentiment, which could improve the teaching of the course in a much effective way.    

\end{enumerate}


\section{Limitations and Scope}

\subsection{Scope}

This research study aims at extracting important points from the comments which are given by the students of Asian Institute of Technology ( AIT ) for the courses provided by the university every semester. To have an overview of students taking the course being satisfied or dissatisfied from the course, whether those comments being positive or negative comments for one specific course using different text mining methods. 

\subsection{Limitations}
May not be able to detect sarcastic comments provided by students.They could be outliers as of data which is not relevant to the evaluation of the course.

\section{Research Outline}

I organize the rest of this dissertation as follows.

In Chapter \ref{ch:literature-review}, I describe the literature review.

In Chapter \ref{ch:methodology}, I propose my methodology.

In Chapter \ref{ch:results}, I present the experimental results.

Finally, in Chapter \ref{ch:conclusion}, I conclude my resarch.

\FloatBarrier

